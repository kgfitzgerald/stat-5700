% Options for packages loaded elsewhere
% Options for packages loaded elsewhere
\PassOptionsToPackage{unicode}{hyperref}
\PassOptionsToPackage{hyphens}{url}
\PassOptionsToPackage{dvipsnames,svgnames,x11names}{xcolor}
%
\documentclass[
  letterpaper,
  DIV=11,
  numbers=noendperiod]{scrartcl}
\usepackage{xcolor}
\usepackage{amsmath,amssymb}
\setcounter{secnumdepth}{-\maxdimen} % remove section numbering
\usepackage{iftex}
\ifPDFTeX
  \usepackage[T1]{fontenc}
  \usepackage[utf8]{inputenc}
  \usepackage{textcomp} % provide euro and other symbols
\else % if luatex or xetex
  \usepackage{unicode-math} % this also loads fontspec
  \defaultfontfeatures{Scale=MatchLowercase}
  \defaultfontfeatures[\rmfamily]{Ligatures=TeX,Scale=1}
\fi
\usepackage{lmodern}
\ifPDFTeX\else
  % xetex/luatex font selection
\fi
% Use upquote if available, for straight quotes in verbatim environments
\IfFileExists{upquote.sty}{\usepackage{upquote}}{}
\IfFileExists{microtype.sty}{% use microtype if available
  \usepackage[]{microtype}
  \UseMicrotypeSet[protrusion]{basicmath} % disable protrusion for tt fonts
}{}
\makeatletter
\@ifundefined{KOMAClassName}{% if non-KOMA class
  \IfFileExists{parskip.sty}{%
    \usepackage{parskip}
  }{% else
    \setlength{\parindent}{0pt}
    \setlength{\parskip}{6pt plus 2pt minus 1pt}}
}{% if KOMA class
  \KOMAoptions{parskip=half}}
\makeatother
% Make \paragraph and \subparagraph free-standing
\makeatletter
\ifx\paragraph\undefined\else
  \let\oldparagraph\paragraph
  \renewcommand{\paragraph}{
    \@ifstar
      \xxxParagraphStar
      \xxxParagraphNoStar
  }
  \newcommand{\xxxParagraphStar}[1]{\oldparagraph*{#1}\mbox{}}
  \newcommand{\xxxParagraphNoStar}[1]{\oldparagraph{#1}\mbox{}}
\fi
\ifx\subparagraph\undefined\else
  \let\oldsubparagraph\subparagraph
  \renewcommand{\subparagraph}{
    \@ifstar
      \xxxSubParagraphStar
      \xxxSubParagraphNoStar
  }
  \newcommand{\xxxSubParagraphStar}[1]{\oldsubparagraph*{#1}\mbox{}}
  \newcommand{\xxxSubParagraphNoStar}[1]{\oldsubparagraph{#1}\mbox{}}
\fi
\makeatother


\usepackage{longtable,booktabs,array}
\usepackage{calc} % for calculating minipage widths
% Correct order of tables after \paragraph or \subparagraph
\usepackage{etoolbox}
\makeatletter
\patchcmd\longtable{\par}{\if@noskipsec\mbox{}\fi\par}{}{}
\makeatother
% Allow footnotes in longtable head/foot
\IfFileExists{footnotehyper.sty}{\usepackage{footnotehyper}}{\usepackage{footnote}}
\makesavenoteenv{longtable}
\usepackage{graphicx}
\makeatletter
\newsavebox\pandoc@box
\newcommand*\pandocbounded[1]{% scales image to fit in text height/width
  \sbox\pandoc@box{#1}%
  \Gscale@div\@tempa{\textheight}{\dimexpr\ht\pandoc@box+\dp\pandoc@box\relax}%
  \Gscale@div\@tempb{\linewidth}{\wd\pandoc@box}%
  \ifdim\@tempb\p@<\@tempa\p@\let\@tempa\@tempb\fi% select the smaller of both
  \ifdim\@tempa\p@<\p@\scalebox{\@tempa}{\usebox\pandoc@box}%
  \else\usebox{\pandoc@box}%
  \fi%
}
% Set default figure placement to htbp
\def\fps@figure{htbp}
\makeatother





\setlength{\emergencystretch}{3em} % prevent overfull lines

\providecommand{\tightlist}{%
  \setlength{\itemsep}{0pt}\setlength{\parskip}{0pt}}



 


\KOMAoption{captions}{tableheading}
\makeatletter
\@ifpackageloaded{caption}{}{\usepackage{caption}}
\AtBeginDocument{%
\ifdefined\contentsname
  \renewcommand*\contentsname{Table of contents}
\else
  \newcommand\contentsname{Table of contents}
\fi
\ifdefined\listfigurename
  \renewcommand*\listfigurename{List of Figures}
\else
  \newcommand\listfigurename{List of Figures}
\fi
\ifdefined\listtablename
  \renewcommand*\listtablename{List of Tables}
\else
  \newcommand\listtablename{List of Tables}
\fi
\ifdefined\figurename
  \renewcommand*\figurename{Figure}
\else
  \newcommand\figurename{Figure}
\fi
\ifdefined\tablename
  \renewcommand*\tablename{Table}
\else
  \newcommand\tablename{Table}
\fi
}
\@ifpackageloaded{float}{}{\usepackage{float}}
\floatstyle{ruled}
\@ifundefined{c@chapter}{\newfloat{codelisting}{h}{lop}}{\newfloat{codelisting}{h}{lop}[chapter]}
\floatname{codelisting}{Listing}
\newcommand*\listoflistings{\listof{codelisting}{List of Listings}}
\makeatother
\makeatletter
\makeatother
\makeatletter
\@ifpackageloaded{caption}{}{\usepackage{caption}}
\@ifpackageloaded{subcaption}{}{\usepackage{subcaption}}
\makeatother
\usepackage{bookmark}
\IfFileExists{xurl.sty}{\usepackage{xurl}}{} % add URL line breaks if available
\urlstyle{same}
\hypersetup{
  pdftitle={HW 05 SOLUTIONS},
  colorlinks=true,
  linkcolor={blue},
  filecolor={Maroon},
  citecolor={Blue},
  urlcolor={Blue},
  pdfcreator={LaTeX via pandoc}}


\title{HW 05 SOLUTIONS}
\author{}
\date{}
\begin{document}
\maketitle


\section{Practice Problems}\label{practice-problems}

\subsection{3.67}\label{section}

\[(0.7)^4 (0.3) = 0.07203\]

\subsection{3.71}\label{section-1}

\begin{enumerate}
\def\labelenumi{(\alph{enumi})}
\item
  \[P(Y > a) = \sum_{y = a + 1}^{\infty} q^{y-1}p = pq^a\sum_{y = 1}^{\infty} q^{y-1} = \frac{pq^a}{1 - q} = q^a\]
\item
  From part (a),
  \[P(Y > a+b \mid Y > a) = \frac{P(Y > a+b)}{P(Y > a)} = \frac{q^{a+b}}{q^a} = q^b\]
\item
  \[P(Y > a+b \mid Y > a) = P(Y > b)\]
\item
  The results in the past are not relevant to a future outcome
  (independent trials).
\end{enumerate}

\subsection{3.73}\label{section-2}

Let \(Y =\) number of accounts audited until the first with substantial
errors is found.

\begin{enumerate}
\def\labelenumi{(\alph{enumi})}
\item
  \[P(Y = 3) = (0.12)(0.9)^2 = 0.009\]
\item
  \[P(Y \geq 3) = P(Y > 2) = (0.9)^2 = 0.81\]
\end{enumerate}

\subsection{3.77}\label{section-3}

\[
\begin{aligned}
P(Y = odd) &= P(Y = 1, 3, 5, \ldots) = P(Y = 2k + 1 \text{ for integers } k = 1, 2, ...)= \sum_{k = 1}^{\infty} q^{2k + 1 - 1}p \\
&= p \sum_{k = 1}^{\infty} q^{2k} = p \sum_{k = 1}^{\infty} (q^2)^{k}\\
&= \frac{p}{(1-q^2)}
\end{aligned}
\]

\subsection{3.93}\label{section-4}

From Ex. 3.92:

\begin{enumerate}
\def\labelenumi{(\alph{enumi})}
\item
  \[P(Y = 5) = \binom{4}{2} (0.9^3)(0.1^2) = 0.04374\]
\item
  \[P(Y \leq 5) = P(Y=3) + P(Y=4) + P(Y=5) = 0.729 + 0.2187 + 0.04374 = 0.99144\]
\end{enumerate}

\subsection{3.97}\label{section-5}

\begin{enumerate}
\def\labelenumi{(\alph{enumi})}
\item
  Geometric probability:\\
  \[(0.8)^2 (0.2) = 0.128\]
\item
  Negative binomial probability:\\
  \[\binom{6}{2} (0.2^3)(0.8^4) = 0.049\]
\item
  The trials are independent and the probability of success is the same
  from trial to trial.
\item
  \[\mu = \frac{3}{0.2} = 15, \quad \sigma^2 = \frac{3(0.8)}{0.2^2} = 60\]
\end{enumerate}

\subsection{3.103}\label{section-6}

Use the hypergeometric distribution with \(N=10, r=4, n=5\):

\[P(Y=0) = \binom{6}{5}\binom{10}{5} = 0.0238\]

\subsection{3.105}\label{section-7}

\begin{enumerate}
\def\labelenumi{(\alph{enumi})}
\item
  \(Y\) follows a hypergeometric distribution. The probability of being
  chosen on a trial is dependent on previous outcomes.
\item
  \[P(Y \geq 2) = P(Y=2) + P(Y=3) = 0.5357 + 0.1786 = 0.7143\]
\item
  \[\mu = 3 \cdot \frac{5}{8} = 1.875, \quad \sigma^2 = 3 \cdot \frac{5}{8} \cdot \frac{3}{8} \cdot \frac{5}{7} = 0.5022, \quad \sigma = 0.7087\]
\end{enumerate}

\subsection{3.121}\label{section-8}

\begin{enumerate}
\def\labelenumi{(\alph{enumi})}
\item
  \[P(Y=4) = \frac{\lambda^4 e^{-\lambda}}{4!} = 0.090\]
\item
  \[P(Y \geq 4) = 1 - P(Y \leq 3) = 1 - 0.857 = 0.143\]
\item
  \[P(Y < 4) = P(Y \leq 3) = 0.857\]
\item
  \[P(Y \geq 4 \mid Y \geq 2) = \frac{P(Y \geq 4)}{P(Y \geq 2)} = \frac{0.143}{0.594} = 0.241\]
\end{enumerate}

\subsection{3.123}\label{section-9}

If \(p(0) = p(1)\), then\\
\[\frac{\lambda^0 e^{-\lambda}}{0!} = \frac{\lambda^1 e^{-\lambda}}{1!}\]\\
So \(\lambda = 1\).

Then\\
\[p(2) = \frac{1^2 e^{-1}}{2!} = 0.1839\] \# Submitted Problems

\subsection{3.70}\label{section-10}

Let \(Y =\) number of holes drilled until a productive well is found.

\begin{enumerate}
\def\labelenumi{(\alph{enumi})}
\item
  \[P(Y=3) = (0.8)^2 (0.2) = 0.128\]
\item
  \[P(Y > 10) = (0.8)^{10} = 0.107\]
\end{enumerate}

\subsection{3.80}\label{section-11}

Let \(Y =\) number of tosses until the first 6 appears, so
\(Y \sim \text{Geometric}(p=1/6)\).

From Ex. 3.77,\\
\[P(\text{B tosses first 6}) = P(Y=2,4,6,\ldots) = 1 - P(Y=1,3,5,\ldots) = 1 - \frac{p}{1-q^2}\]

With \(p=1/6\), this becomes \(5/11\).

Then\\
\[P(Y=4 \mid \text{B tosses first 6}) = \frac{(5/6)^3 (1/6)}{5/11} = 275/1296\]

\subsection{3.92}\label{section-12}

Let \(Y =\) number of trials until the first non-defective engine is
found. \(Y \sim \text{Geometric}(p=0.9)\).

\[P(Y=2) = (0.1)(0.9) = 0.09\]

\subsection{3.94}\label{section-13}

\begin{enumerate}
\def\labelenumi{(\alph{enumi})}
\item
  \[\mu = \frac{1}{0.9} = 1.11, \quad \sigma^2 = \frac{0.1}{0.9^2} = 0.1234\]
\item
  \[\mu = \frac{3}{0.9} = 3.33, \quad \sigma^2 = \frac{3(0.1)}{0.9^2} = 0.3704\]
\end{enumerate}

\subsection{3.106}\label{section-14}

Using results from Ex. 3.103:

\[E(50Y) = 50E(Y) = 50 \cdot 2 = 100\]\\
\[V(50Y) = 2500V(Y) = 2500 \cdot 0.6667 = 1666.67\]

\subsection{3.108}\label{section-15}

Use \(P(\text{at least one defective}) = 1 - P(\text{none defective})\).

Want \(1 - \frac{\binom{3}{0}\binom{17}{n}}{\binom{20}{n}} \geq 0.8\)

For \(n=8\):\\
\[P(\text{none defective}) = \frac{{17 \choose 8}}}{{20 \choose 8}}} = 0.193
\]

\subsection{3.120}\label{section-16}

There are \(N\) animals. After marking \(k\), we draw a second sample of
3.

Number of ways to find exactly one marked animal:\\
\[P(Y=1) = \frac{{k \choose 1}} {{N-k \choose 2}}}{{N \choose 3}}}
\]

For \(k=4\):\\
\[P(Y=1) = \frac{{4 \choose 1}} {{N-4 \choose 2}}}{{N \choose 3}}}
\] This is maximized at \(N=11\) or \(N=12\), where \(P=0.503\).

\subsection{3.128}\label{section-17}

Over one minute, \(Y \sim \text{Poisson}(\lambda = 80/60 = 4/3)\).

\[P(Y \geq 1) = 1 - P(Y=0) = 1 - e^{-4/3} = 0.7364\]

\subsection{3.138}\label{section-18}

\[ 
\begin{aligned} 
E(Y(Y-1)) &= \sum_{y = 0}^{\infty} \frac{y(y-1)\lambda^ye^{-\lambda}}{y!} \\ 
&= \sum_{y = 2}^{\infty} \frac{y(y-1)\lambda^ye^{-\lambda}}{y!} \\ 
&= e^{-\lambda}\sum_{y = 2}^{\infty} \frac{\lambda^y}{(y-2)!}\frac{\lambda^2}{\lambda^2} \\ 
&= e^{-\lambda}\lambda^2\sum_{y - 2 = 0}^{\infty} \frac{\lambda^{y - 2}}{(y-2)!} \\ 
&= e^{-\lambda}\lambda^2e^{\lambda} \\ 
&= \lambda^2 
\end{aligned} 
\] Then\\
\[E[Y(Y-1)] = E(Y^2) - E(Y) = \lambda^2 \implies E(Y^2) = \lambda^2 + \lambda\]

And
\(V(X) = E(Y^2) - [E(Y)]^2 = \lambda^2 + \lambda - \lambda^2 = \lambda\)




\end{document}
